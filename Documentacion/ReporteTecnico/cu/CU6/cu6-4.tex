	\begin{UseCase}{CU6.4}{Consultar Término}{
		Este caso de uso permite al analista consultar la información de un término del glosario.
	}
		\UCitem{Versión}{\color{Gray}0.1}
		\UCitem{Actor}{\hyperlink{jefe}{Líder de Análisis}, \hyperlink{analista}{Analista}}
		\UCitem{Propósito}{Consultar la información de un término del glosario.}
		\UCitem{Entradas}{Ninguna}
		\UCitem{Salidas}{\begin{itemize}
				\item \cdtRef{terminoGLSEntidad:nombreTGLS}{Nombre:} Lo obtiene el sistema.
				\item \cdtRef{terminoGLSEntidad:descripcionTGLS}{Descripción:} Lo obtiene el sistema.
		\end{itemize}}
		\UCitem{Destino}{Pantalla}
		\UCitem{Precondiciones}{Ninguna}
		\UCitem{Postcondiciones}{Ninguna}
		\UCitem{Errores}{\begin{itemize}
		\item \cdtIdRef{MSG12}{Ha ocurrido un error}: Se muestra en la pantalla \IUref{IU11}{Gestionar Términos del glosario} cuando el término que se desea consultar no existe.
		\end{itemize}
		}
		\UCitem{Tipo}{Secundario, extiende del caso de uso \UCref{CU6}{Gestionar Términos del glosario}.}
	\end{UseCase}
%--------------------------------------
	\begin{UCtrayectoria}
		\UCpaso[\UCactor] Solicita eliminar un término del glosario oprimiendo el botón \raisebox{-1mm}{\includegraphics[height=11pt]{images/Iconos/consultar}} del registro que desea consultar de la pantalla \IUref{IU11}{Gestionar Términos de glosario} o la liga correspondiente a un término en la pantalla \IUref{IU6.3}{Consultar caso de uso}.
		\UCpaso[\UCsist] Obtiene la información del término seleccionado. \Trayref{CT-A}
		\UCpaso[\UCsist] Muestra la pantalla \IUref{IU11.3}{Consultar Término} con la información del término.
		\UCpaso[\UCactor] Solicita finalizar oprimiendo el botón \IUref{Regresar} de la pantalla \IUref{IU11.3}{Consultar Término}.
	\end{UCtrayectoria}		
%--------------------------------------
	
	\begin{UCtrayectoriaA}{CT-A}{El término que se desea consultar no existe.}
		\UCpaso[\UCsist] Muestra la pantalla \IUref{IU11}{Gestionar Términos del glosario} con el mensaje \cdtIdRef{MSG12}{Ha ocurrido un error}.
	\end{UCtrayectoriaA}
