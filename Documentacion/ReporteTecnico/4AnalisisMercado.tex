\chapter{Análisis de mercado}

En este apartado se demuestra la viabilidad comercial del trabajo terminal TESSERACT en México, así mismo se realiza un estudio en donde se determina el campo en donde un sistema con las características del generador de documento de casos de uso podría generar un mayor impacto y aceptación por parte de los equipos de construcción de software. Cabe resaltar que TESSERACT no pretende ser comercializado por el momento.

%---------------------------------------------------------
\section{Situación actual  y evolución del mercado}

El software es un elemento consustancial a la economía moderna. Se usa en casi todos los productos manufacturados y en los servicios; aun cuando hay empresas especializadas en su desarrollo que parecen constituir una industria distinta, son apenas un segmento de una más compleja; esta clase de industria no refleja en realidad su verdadero desarrollo, ya que la elaboración de programas de cómputo figura en casi
todas las industrias y es, de hecho, factor de éxito de todos los sectores de la economía. 

El desarrollo de software, es uno de los sectores tecnológicos más competitivos y no es algo nuevo, ya que durante muchos años lo ha sido, sin embargo ha tenido una evolución constante en lo que se refiere a las metodologías o bien, las formas en las cuales se realiza la planeación para el diseño del software, básicamente con el objetivo de mejorar, optimizar procesos y ofrecer una mejor calidad.

En el campo del desarrollo de software, existen dos grupos de metodologías, las denominadas tradicionales (formales) y las ágiles.
Las primeras son un tanto rígidas, exigen una documentación exhaustiva y se centran en
cumplir con el plan del proyecto definido totalmente en la fase inicial del desarrollo del mismo; mientras que la segunda enfatiza el esfuerzo en la capacidad de respuesta a los cambios, las habilidades del equipo y mantener una buena relación con el usuario.
Ambas propuestas tienen sus propias ventajas y desventajas; de cualquier manera, las
metodologías de desarrollo nos dicen el ¿Qué hacer? más no el ¿Cómo hacer?, esto significa que la metodología que elijamos, debe ser adaptada al contexto del proyecto, teniendo en cuenta los recursos técnicos y humanos; tiempo de desarrollo y tipo de sistema.


\subsection{Industria Mexicana del Software}
Antes de describir el perfil de las empresas desarrolladoras de software en México, es importante destacar que los diversos análisis que hasta la fecha se han realizado con respecto al panorama de este sector no resultan aún generalizables a toda la industria, ya que cada estudio analiza sólo un subconjunto del total de empresas, por lo tanto se hace la aclaración que lo aquí se presenta son datos representativos, y no necesariamente significa que sean generalizables.

Localización Geográfica de las Empresas Participantes
Las empresas participantes en el estudio se localizan en 11 de los 32 estados de la República Mexicana, presentando la siguiente distribución: 2.9\% Chihuahua, 1.5\%  en Coahuila, 44.1\%  en la Ciudad de México, 11.8\%  en Durango, 2.9\%  en el Estado de México, 1.5\%  en Guanajuato, 2.9\%  en Jalisco, 2.9\%  en Michoacán, 2.9\%  en Morelos, 23.5\%  en Nuevo León y 2.9\%  en Querétaro. Esta concentración es similar a la de otros estudios realizados para este sector en México [1, 2].

Número de Empresas Desarrolladoras de Software en México
La respuesta a esta pregunta no tiene una cifra exacta. De acuerdo con estimaciones realizadas por ESANE consultores [2] sobre del número total de empleados y empresas de la Industria del Software en México, el número aproximado de empresas de la industria mexicana del software podría ser del orden de 1,500 empresas.

Tamaño de las Empresas
El estudio revela que el 85.29\% de las empresas del sector de la Industria Mexicana del Software son de tamaño micro (54.41\%) y pequeño (30.88\%), el 5.8\% mediana, y tan sólo el 8.82\% son de tamaño grande (con un número de empleados mayor a 100).