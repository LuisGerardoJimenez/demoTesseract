	\begin{UseCase}{CU7.1.2}{Modificar Atributo}{
		Este caso de uso permite al analista modificar la información de un atributo.
	}
		\UCitem{Versión}{\color{Gray}0.1}
		\UCitem{Actor}{\hyperlink{jefe}{Líder de Análisis}, \hyperlink{analista}{Analista}}
		\UCitem{Propósito}{Modificar la información de atributo.}
		\UCitem{Entradas}{
		\begin{itemize}
			\item \cdtRef{atributoEntidad:nombreATR}{Nombre:} Se escribe desde el teclado.
			\item \cdtRef{atributoEntidad:descripcionATR}{Descripción:} Se escribe desde el teclado.
			\item \cdtRef{atributoEntidad:obligatorioATR}{Obligatorio:} Se selecciona de una lista.
			\item \cdtRef{atributoEntidad:longitudATR}{Descripción:} Se escribe desde el teclado.
			\item \hyperlink{tTipoDatoP}{Tipo de Dato:} Se selecciona de una lista.
			\item Formato de archivo: Se escribe desde el teclado.
			\item Tamaño de archivo: Se escribe desde el teclado.
			\item Unidad: Se selecciona de una lista.
			\item Otro tipo de dato: Se escribe desde el teclado.
		\end{itemize}	
		}
		\UCitem{Salidas}{\begin{itemize}
			\item \cdtRef{atributoEntidad:nombreATR}{Nombre:} Lo obtiene el sistema.
			\item \cdtRef{atributoEntidad:descripcionATR}{Descripción:} Lo obtiene el sistema.
			\item \cdtRef{atributoEntidad:obligatorioATR}{Obligatorio:} Lo obtiene el sistema.
			\item \cdtRef{atributoEntidad:longitudATR}{Descripción:} Lo obtiene el sistema.
			\item \hyperlink{tTipoDatoP}{Tipo de Dato:} Lo obtiene el sistema.
			\item Formato de archivo: Lo obtiene el sistema.
			\item Tamaño de archivo: Lo obtiene el sistema.
			\item Unidad: Lo obtiene el sistema.
			\item Otro tipo de dato: Lo obtiene el sistema.
		\end{itemize}
		}
		\UCitem{Destino}{Pantalla}
		\UCitem{Precondiciones}{Ninguna}
		\UCitem{Postcondiciones}{Ninguna}
		\UCitem{Errores}{\begin{itemize}
		\item \cdtIdRef{MSG4}{Dato obligatorio}: Se muestra en la pantalla \IUref{IU12.1.2}{Modificar Atributo} cuando no se ha ingresado un dato marcado como obligatorio.
		\item \cdtIdRef{MSG29}{Formato incorrecto}: Se muestra en la pantalla \IUref{IU12.1.2}{Modificar Atributo} cuando el tipo de dato ingresado no cumple con el tipo de dato solicitado en el campo.
		\item \cdtIdRef{MSG6}{Longitud inválida}: Se muestra en la pantalla \IUref{IU12.1.2}{Modificar Atributo} cuando se ha excedido la longitud de alguno de los campos.
		\item \cdtIdRef{MSG7}{Registro repetido}: Se muestra en la pantalla \IUref{IU12.1.2}{Modificar Atributo} cuando se registre un atributo con un nombre que ya se encuentra registrado en el sistema.
		\item \cdtIdRef{MSG18}{Caracteres inválidos}: Se muestra en la pantalla \IUref{IU12.1.2}{Modificar Atributo} cuando el nombre del atributo contiene un carácter no válido
		\end{itemize}
		}
		\UCitem{Tipo}{Secundario, extiende de los casos de uso \UCref{CU7.1}{Registrar Entidad} y \UCref{CU7.2}{Modificar Entidad}.}
	\end{UseCase}
%--------------------------------------
	\begin{UCtrayectoria}
		\UCpaso[\UCactor] Solicita modificar un atributo oprimiendo el botón \editar de algún registro existente en la pantalla \IUref{IU12.1}{Registrar Entidad} o \IUref{IU12.2}{Modificar Entidad}.
		\UCpaso[\UCsist] Obtiene la información del atributo.
		\UCpaso[\UCsist] Muestra la pantalla \IUref{IU12.1.2}{Modificar Atributo} con la información encontrada.
		\UCpaso[\UCactor] Ingresa la información solicitada en la pantalla. \label{CU7.1.2-P4}
		\UCpaso[\UCactor] Solicita guardar la información de atributo oprimiendo el botón \IUbutton{Aceptar} de la pantalla \IUref{IU12.1.2}{Modificar Atributo}. \hyperlink{CU7-1-2:TAA}{[Trayectoria A]}
		\UCpaso[\UCsist] Verifica que el actor ingrese todos los campos obligatorios con base en la regla de negocio \BRref{RN8}{Datos obligatorios}. \hyperlink{CU7-1-2:TAB}{[Trayectoria B]}
		\UCpaso[\UCsist] Verificar que los datos ingresados cumpla con la longitud correcta, con base en la regla de negocio \BRref{RN37}{Longitud de datos}. \hyperlink{CU7-1-2:TAC}{[Trayectoria C]}
		\UCpaso[\UCsist] Verifica que los datos ingresados cumplan con el formato requerido, con base en la regla de negocio \BRref{RN7}{Información correcta}. \hyperlink{CU7-1-2:TAD}{[Trayectoria D]}
		\UCpaso[\UCsist] Verifica que el nombre del atributo no se encuentre registrado en el sistema con base en la regla de negocio \BRref{RN6}{Unicidad de nombres}. \hyperlink{CU7-1-2:TAE}{[Trayectoria E]}
		\UCpaso[\UCsist] Modifica la información del atributo y actualiza la tabla de la pantalla \IUref{IU12.1}{Registrar Entidad} o \IUref{IU12.2}{Modificar Entidad}.
	\end{UCtrayectoria}		
%--------------------------------------
	
\hypertarget{CU7-1-2:TAA}{\textbf{Trayectoria alternativa A}}\\
\noindent \textbf{Condición:} El actor desea cancelar la operación.
\begin{enumerate}
	\UCpaso[\UCactor] Solicita cancelar la operación oprimiendo el botón \IUbutton{Cancelar} de la pantalla \IUref{IU12.1.2}{Modificar Atributo}
	\UCpaso[\UCsist] Muestra la pantalla \IUref{IU12.1}{Registrar Entidad} ó \IUref{IU12.2}{Modificar Entidad}.
	\item[- -] - - {\em {Fin del caso de uso}}.%
\end{enumerate}
%--------------------------------------	
\hypertarget{CU7-1-2:TAB}{\textbf{Trayectoria alternativa B}}\\
\noindent \textbf{Condición:} El actor no ingresó algún dato marcado como obligatorio.
\begin{enumerate}
	\UCpaso[\UCsist] Muestra el mensaje \cdtIdRef{MSG4}{Dato obligatorio} señalando el campo que presenta el error en la pantalla \IUref{IU12.1.2}{Modificar Atributo}.
	\UCpaso Regresa al paso \ref{CU7.1.2-P4} de la trayectoria principal.
	\item[- -] - - {\em {Fin de la trayectoria}}.%
\end{enumerate}
%--------------------------------------
\hypertarget{CU7-1-2:TAC}{\textbf{Trayectoria alternativa C}}\\
\noindent \textbf{Condición:} El actor ingresó un dato con un número de caracteres fuera del rango permitido.
\begin{enumerate}
	\UCpaso[\UCsist] Muestra el mensaje \cdtIdRef{MSG6}{Longitud inválida} señalando el campo que presenta el error en la pantalla \IUref{IU12.1.2}{Modificar Atributo}.
	\UCpaso Regresa al paso \ref{CU7.1.2-P4} de la trayectoria principal.
	\item[- -] - - {\em {Fin de la trayectoria}}.%
\end{enumerate}
%-------------------------------------
\hypertarget{CU7-1-2:TAD}{\textbf{Trayectoria alternativa D}}\\
\noindent \textbf{Condición:} El actor ingresó un dato con un formato de dato incorrecto.
\begin{enumerate}
	\UCpaso[\UCsist] Muestra el mensaje \cdtIdRef{MSG29}{Formato incorrecto} señalando el campo que presenta el error en la pantalla \IUref{IU12.1.2}{Modificar Atributo}.
	\UCpaso Regresa al paso \ref{CU7.1.2-P4} de la trayectoria principal.
	\item[- -] - - {\em {Fin de la trayectoria}}.
\end{enumerate}
%-------------------------------------
\hypertarget{CU7-1-2:TAE}{\textbf{Trayectoria alternativa E}}\\
\noindent \textbf{Condición:} El actor ingresó un atributo que ya existe dentro de la entidad.
\begin{enumerate}
	\UCpaso[\UCsist] Muestra el mensaje \cdtIdRef{MSG7}{Registro repetido} señalando el campo que presenta la duplicidad en la pantalla \IUref{IU12.1.2}{Modificar Atributo}.
	\UCpaso Regresa al paso \ref{CU7.1.2-P4} de la trayectoria principal.
	\item[- -] - - {\em {Fin de la trayectoria}}.
\end{enumerate}

