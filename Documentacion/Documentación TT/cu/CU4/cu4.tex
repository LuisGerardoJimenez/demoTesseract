	\begin{UseCase}{CU4}{Gestionar Proyectos de Colaborador}{
		Este caso de uso permite al actor visualizar los proyectos en los que se encuentra participando, además sirve como punto de acceso para gestionar: módulos, términos del glosario, entidades, reglas de negocio, mensajes y actores, así como para descargar el documento de análisis y en caso de ser líder, elegir a los colaboradores.
	}
	\UCitem{Versión}{\color{Gray}0.1}
	\UCitem{Actor}{\hyperlink{jefe}{Líder de análisis}, \hyperlink{admin}{Administrador}}
	\UCitem{Propósito}{Visualizar los proyectos a los que se encuentra asociado, así como entrar a cada uno de ellos para realizar las actividades correspondientes}
	\UCitem{Entradas}{Ninguna}
	\UCitem{Salidas}{\begin{itemize}
			\item : \cdtRef{proyectoEntidad}{Proyecto}: Tabla que muestra \cdtRef{proyectoEntidad:claveProyecto}{Clave}, \cdtRef{proyectoEntidad:nombreProyecto}{Nombre} y el \cdtRef{proyectoEntidad:liderProyecto}{Líder del Proyecto} de todos los registros de los proyectos.
			\item  \cdtIdRef{MSG2}{No existe información}: Se muestra en la pantalla \IUref{IU4}{Gestionar proyectos de colaborador} cuando el actor no se encuentra asociado a ningún proyecto.
	\end{itemize}}
	\UCitem{Destino}{Pantalla}
	\UCitem{Precondiciones}{Ninguna}
	\UCitem{Postcondiciones}{Ninguna}
	\UCitem{Errores}{Ninguno}
	\UCitem{Tipo}{Caso de uso primario}
\end{UseCase}
%--------------------------------------
\begin{UCtrayectoria}
	\UCpaso[\UCactor] Solicita gestionar los proyectos presionando la opción ''Proyectos'' del menú \IUref{MN2}{Menú de Colaborador}.
	\UCpaso[\UCsist] Obtiene la información de los proyectos en los que el actor se encuentra colaborando. \Trayref{GPC-A}
	\UCpaso[\UCsist] Muestra la información del personal en la pantalla \IUref{IU4}{Gestionar Proyectos de colaborador}.
	\UCpaso[\UCsist] Muestra el botón \raisebox{-1mm}{\includegraphics[height=11pt]{images/Iconos/Colaboradores}} para cada proyecto en el que el actor sea líder.
	\UCpaso[\UCactor] Gestiona los proyectos a través de las botones mostrados en la columna ''Acciones''. \label{CU4-P5}
\end{UCtrayectoria}		
%--------------------------------------
\begin{UCtrayectoriaA}{GPC-A}{El actor no se encuentra colaborando en ningún proyecto.}
	\UCpaso[\UCsist] Muestra el mensaje \cdtIdRef{MSG2}{No existe información} en la pantalla \IUref{IU4}{Gestionar Proyectos de Colaborador} para indicar que no hay registros de proyectos para mostrar.
\end{UCtrayectoriaA}

%--------------------------------------

\subsubsection{Puntos de extensión}

\UCExtenssionPoint{El actor requiere gestionar los módulos de un proyecto.}{Paso \ref{CU4-P5} de la trayectoria principal.}{\UCref{CU5}{Gestionar Módulos}}
\UCExtenssionPoint{El actor requiere gestionar el glosario de un proyectos.}{Paso \ref{CU4-P5} de la trayectoria principal.}{\UCref{CU10}{Gestoinar Términos}}
\UCExtenssionPoint{El actor requiere gestionar las entidades de un proyecto.}{Paso \ref{CU4-P5} de la trayectoria principal.}{\UCref{CU11}{Gestionar Entidades}}
\UCExtenssionPoint{El actor requiere gestionar las reglas de negocio de un proyecto.}{Paso \ref{CU4-P5} de la trayectoria principal.}{\UCref{CU8}{Gestionar reglas de negocio}}
\UCExtenssionPoint{El actor requiere gestionar los mensajes de un proyecto.}{Paso \ref{CU4-P5} de la trayectoria principal.}{\UCref{CU9}{Gestionar Mensajes}}
\UCExtenssionPoint{El actor requiere gestionar los actores de un proyecto.}{Paso \ref{CU4-P5} de la trayectoria principal.}{\UCref{CU9}{Gestionar Actores}}
\UCExtenssionPoint{El actor requiere descargar el documento de análisis de un proyecto.}{Paso \ref{CU4-P5} de la trayectoria principal.}{\UCref{CU3.3}{Descargar Documento}}
\UCExtenssionPoint{El actor requiere elegir a los colaboradores de un proyecto.}{Paso \ref{CU4-P5} de la trayectoria principal.}{\UCref{CU3.3}{Elegir Colaboradores}}
