	\begin{UseCase}{CU3.1}{Registrar persona}{
	Permite registrar la información de una persona que podrá ser elegida como colaborador de un proyecto.
	}
		\UCitem{Versión}{\color{Gray}0.1}
		\UCitem{Actor}{\hyperlink{admin}{Administrador}}
		\UCitem{Propósito}{Registrar la información de una nueva persona para que pueda ser asociada como colaborador de uno o varios proyetos.}
		\UCitem{Entradas}{
		\begin{itemize}
			\item \cdtRef{colaboradorEntidad:curpColaborador}{CURP}: Se escribe desde el teclado.
			\item \cdtRef{colaboradorEntidad:nombreColaborador}{Nombre}: Se escribe desde el teclado.
			\item \cdtRef{colaboradorEntidad:pApellidoColaborador}{Primer Apellido}: Se escribe desde el teclado.
			\item \cdtRef{colaboradorEntidad:sApellidoColaborador}{Segundo Apellido}: Se escribe desde el teclado.
			\item \cdtRef{colaboradorEntidad:correoColaborador}{Correo electrónico}: Se escribe desde el teclado.
			\item \cdtRef{colaboradorEntidad:passColaborador}{Contraseña}: Se escribe desde el teclado.
		\end{itemize}	
		}
		\UCitem{Salidas}{\begin{itemize}
				\item \cdtIdRef{MSG1}{Operación exitosa}: Se muestra en la pantalla \IUref{IU3}{Gestionar Personal} para indicar que el registro fue exitoso.
		\end{itemize}}
		\UCitem{Destino}{Pantalla}
		\UCitem{Precondiciones}{Ninguna}
		\UCitem{Postcondiciones}{
		\begin{itemize}
			\item Se registrará una persona en el sistema.
			\item La persona registrada se podrá elegir para participar en algún proyecto.
		\end{itemize}
		}
		\UCitem{Errores}{\begin{itemize}
		\item \cdtIdRef{MSG4}{Dato obligatorio}: Se muestra en la pantalla \IUref{IU3.1}{Registrar Persona} cuando no se ha ingresado un dato marcado como obligatorio.
		\item \cdtIdRef{MSG29}{Formato incorrecto}: Se muestra en la pantalla \IUref{IU3.1}{Registrar Persona} cuando el tipo de dato ingresado no cumple con el tipo de dato solicitado en el campo.
		\item \cdtIdRef{MSG28}{Longitud de CURP inválida}: Se muestra en la pantalla \IUref{IU3.1}{Registrar Persona} cuando la CURP ingresada no cumple con la longitud especificada.
		\item \cdtIdRef{MSG30}{CURP inválida}: Se muestra en la pantalla \IUref{IU3.1}{Registrar Persona} cuando la CURP ingresada no cumple con el formato definido.
		\item \cdtIdRef{MSG6}{Longitud inválida}: Se muestra en la pantalla \IUref{IU3.1}{Registrar Persona} cuando se ha excedido la longitud de alguno de los campos.
		\item \cdtIdRef{MSG7}{Registro repetido}: Se muestra en la pantalla \IUref{IU3.1}{Registrar Persona} cuando se registre una persona con una CURP que ya se encuentra registrada en el sistema.
		\end{itemize}
		}
		\UCitem{Tipo}{Secundario, extiende del caso de uso \UCref{CU3}{Gestionar Personal}.}
	\end{UseCase}
%--------------------------------------
	\begin{UCtrayectoria}
		\UCpaso[\UCactor] Solicita registrar una persona oprimiendo el botón \IUbutton{Registrar} de la pantalla \IUref{IU3}{Gestionar Personal}.
		\UCpaso[\UCsist] Muestra la pantalla \IUref{IU3.1}{Registrar Persona}.
		\UCpaso[\UCactor] Ingresa la información solicitada en la pantalla. \label{CU3.1-P3}
		\UCpaso[\UCactor] Solicita guardar a la persona oprimiendo el botón \IUbutton{Aceptar} de la pantalla \IUref{IU3.1}{Registrar Persona}. \hyperlink{CU3-1:TAA}{[Trayectoria A]}
		\UCpaso[\UCsist] Verifica que el actor ingrese todos los campos obligatorios con base en la regla de negocio \BRref{RN8}{Datos obligatorios}. \hyperlink{CU3-1:TAB}{[Trayectoria B]}
		\UCpaso[\UCsist] Verificar que los datos ingresados cumpla con la longitud correcta, con base en la regla de negocio \BRref{RN37}{Longitud de datos}. \hyperlink{CU3-1:TAC}{[Trayectoria C]} \hyperlink{CU3-1:TAD}{[Trayectoria D]}
		\UCpaso[\UCsist] Verifica que los datos ingresados cumplan con el formato requerido, con base en la regla de negocio \BRref{RN7}{Información correcta}. \hyperlink{CU3-1:TAE}{[Trayectoria E]} \hyperlink{CU3-1:TAF}{[Trayectoria F]}
		\UCpaso[\UCsist] Verifica que la CURP de persona no se encuentre registrado en el sistema con base en la regla de negocio \BRref{RN33}{Unicidad de la CURP}. \hyperlink{CU3-1:TAG}{[Trayectoria G]}
		\UCpaso[\UCsist] Verifica que el correo de persona no se encuentre registrado en el sistema con base en la regla de negocio \BRref{RN36}{Unicidad de correos}. \hyperlink{CU3-1:TAH}{[Trayectoria H]}
		\UCpaso[\UCsist] Registra la información de la persona en el sistema
		\UCpaso[\UCsist] Envía un correo con el mensaje \cdtIdRef{MSG25}{Datos de sesión} a la cuenta de correo electrónico proporcionada por el actor.
		\UCpaso[\UCsist] Muestra el mensaje \cdtIdRef{MSG1}{Operación exitosa} en la pantalla \IUref{IU3}{Gestionar Personal} para indicar al actor que el registro se ha realizado exitosamente.
	\end{UCtrayectoria}		
%--------------------------------------
\hypertarget{CU3-1:TAA}{\textbf{Trayectoria alternativa A}}\\
\noindent \textbf{Condición:} El actor desea cancelar la operación.
\begin{enumerate}
	\UCpaso[\UCactor] Solicita cancelar la operación oprimiendo el botón \IUbutton{Cancelar} de la pantalla \IUref{IU3.1}{Registrar Persona}.
	\UCpaso[\UCsist] Muestra la pantalla \IUref{IU3}{Gestionar Personal}
	\item[- -] - - {\em {Fin del caso de uso}}.%
\end{enumerate}
%--------------------------------------	
\hypertarget{CU3-1:TAB}{\textbf{Trayectoria alternativa B}}\\
\noindent \textbf{Condición:} El actor no ingresó algún dato marcado como obligatorio.
\begin{enumerate}
	\UCpaso[\UCsist] Muestra el mensaje \cdtIdRef{MSG4}{Dato obligatorio} señalando el campo que presenta el error en la pantalla \IUref{IU3.1}{Registrar Personal}.
	\UCpaso Regresa al paso \ref{CU3.1-P3} de la trayectoria principal.
	\item[- -] - - {\em {Fin de la trayectoria}}.%
\end{enumerate}
%--------------------------------------
\hypertarget{CU3-1:TAC}{\textbf{Trayectoria alternativa C}}\\
\noindent \textbf{Condición:} El actor ingresó una CURP con una longitud incorrecta.
\begin{enumerate}
	\UCpaso[\UCsist] Muestra el mensaje \cdtIdRef{MSG28}{Longitud de CURP inválida} y señala el campo que presenta el error en la pantalla \IUref{IU3.1}{Registrar Personal}.
	\UCpaso Regresa al paso \ref{CU3.1-P3} de la trayectoria principal.
	\item[- -] - - {\em {Fin de la trayectoria}}.%
\end{enumerate}
%--------------------------------------
\hypertarget{CU3-1:TAD}{\textbf{Trayectoria alternativa D}}\\
\noindent \textbf{Condición:} El actor ingresó un dato con un número de caracteres fuera del rango permitido.
\begin{enumerate}
	\UCpaso[\UCsist] Muestra el mensaje \cdtIdRef{MSG6}{Longitud inválida} señalando el campo que presenta el error en la pantalla \IUref{IU3.1}{Registrar Personal}.
	\UCpaso Regresa al paso \ref{CU3.1-P3} de la trayectoria principal.
	\item[- -] - - {\em {Fin de la trayectoria}}.%
\end{enumerate}
%--------------------------------------	
\hypertarget{CU3-1:TAE}{\textbf{Trayectoria alternativa E}}\\
\noindent \textbf{Condición:} El actor ingresó una CURP inválida.
\begin{enumerate}
	\UCpaso[\UCsist] Muestra el mensaje \cdtIdRef{MSG30}{CURP inválida} señalando el campo que presenta el error en la pantalla \IUref{IU3.1}{Registrar Personal}.
	\UCpaso Regresa al paso \ref{CU3.1-P3} de la trayectoria principal.
	\item[- -] - - {\em {Fin de la trayectoria}}.
\end{enumerate}
%--------------------------------------
\hypertarget{CU3-1:TAF}{\textbf{Trayectoria alternativa F}}\\
\noindent \textbf{Condición:} El actor ingresó un dato con un formato incorrecto.
\begin{enumerate}
	\UCpaso[\UCsist] Muestra el mensaje \cdtIdRef{MSG29}{Formato incorrecto} señalando el campo que presenta el error en la pantalla \IUref{IU3.1}{Registrar Persona}.
	\UCpaso Regresa al paso \ref{CU3.1-P3} de la trayectoria principal.
	\item[- -] - - {\em {Fin de la trayectoria}}.
\end{enumerate}
%--------------------------------------
\hypertarget{CU3-1:TAG}{\textbf{Trayectoria alternativa G}}\\
\noindent \textbf{Condición:} El actor ingresó una CURP que ya existe dentro del sistema.
\begin{enumerate}
	\UCpaso[\UCsist] Muestra el mensaje \cdtIdRef{MSG7}{Registro repetido} señalando el campo que presenta la duplicidad en la pantalla \IUref{IU3.1}{Registrar Personal}.
	\UCpaso Regresa al paso \ref{CU3.1-P3} de la trayectoria principal.
	\item[- -] - - {\em {Fin de la trayectoria}}.
\end{enumerate}
%--------------------------------------
\hypertarget{CU3-1:TAH}{\textbf{Trayectoria alternativa H}}\\
\noindent \textbf{Condición:} El actor ingresó una correo electrónico repetido.
\begin{enumerate}
	\UCpaso[\UCsist] Muestra el mensaje \cdtIdRef{MSG7}{Registro repetido} señalando el campo que presenta la duplicidad en la pantalla \IUref{IU3.1}{Registrar Personal}.
	\UCpaso Regresa al paso \ref{CU3.1-P3} de la trayectoria principal.
	\item[- -] - - {\em {Fin de la trayectoria}}.
\end{enumerate}

