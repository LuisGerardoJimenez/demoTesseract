	\begin{UseCase}{CU2.1}{Registrar proyecto}{
		Este caso de uso permite al actor registrar la información de un proyecto en el sistema.
	}
		\UCitem{Versión}{\color{Gray}0.1}
		\UCitem{Actor}{\hyperlink{admin}{Administrador}}
		\UCitem{Propósito}{Proporcionar al actor un mecanismo para llevar el control de los proyectos.}
		\UCitem{Entradas}{
		\begin{itemize}
			\item \cdtRef{proyectoEntidad:claveProyecto}{Clave:} Se escribe desde el teclado.
			\item \cdtRef{proyectoEntidad:nombreProyecto}{Nombre:} Se escribe desde el teclado.
			\item \cdtRef{proyectoEntidad:fechaIProyecto}{Fecha de inicio:} Se selecciona de un calendario.
			\item \cdtRef{proyectoEntidad:fechaFinProyecto}{Fecha de término:} Se selecciona de un calendario.
			\item \cdtRef{proyectoEntidad:fechaIPProyecto}{Fecha de inicio programada:} Se selecciona de un calendario.
			\item \cdtRef{proyectoEntidad:fechaFinPProyecto}{Fecha de término programada:} Se selecciona de un calendario.
			\item \cdtRef{proyectoEntidad:liderProyecto}{Líder del Proyecto:} Se selecciona de una lista.
			\item \cdtRef{proyectoEntidad:descripcionProyecto}{Descripción:} Se escribe desde el teclado.
			\item \cdtRef{proyectoEntidad:contraparteProyecto}{Contraparte} Se escribe desde el teclado.
			\item \cdtRef{proyectoEntidad:presupuestoProyecto}{Presupuesto:} Se escribe desde el teclado.
			\item \hyperlink{tEdoProy}{Estado del Proyecto:} Se selecciona de un lista.
		\end{itemize}	
		}
		\UCitem{Salidas}{\begin{itemize}
				\item \cdtIdRef{MSG1}{Operación exitosa}: Se muestra en la pantalla \IUref{IU2}{Gestionar proyectos de Administrador} para indicar que el registro fue exitoso.
		\end{itemize}}
		\UCitem{Destino}{Pantalla}
		\UCitem{Precondiciones}{
		\begin{itemize}
			\item Que exista al menos un colaborador registrado.
			\item Que exista información referente a los estados del proyecto.
		\end{itemize}
		}
		\UCitem{Postcondiciones}{
		\begin{itemize}
			\item Se registrará un proyecto en el sistema.
			\item Se podrán gestionar los Términos del glosario, Entidades, Reglas de negocio, Mensajes y Actores.
		\end{itemize}
		}
		\UCitem{Errores}{\begin{itemize}
		\item \cdtIdRef{MSG4}{Dato obligatorio}: Se muestra en la pantalla \IUref{IU2.1}{Registrar proyecto} cuando no se ha ingresado un dato marcado como obligatorio.
		\item \cdtIdRef{MSG5}{Dato incorrecto}: Se muestra en la pantalla \IUref{IU2.1}{Registrar proyecto} cuando el tipo de dato ingresado no cumple con el tipo de dato solicitado en
		el campo.
		\item \cdtIdRef{MSG6}{Longitud inválida}: Se muestra en la pantalla \IUref{IU2.1}{Registrar proyecto} cuando se ha excedido la longitud de alguno de los campos.
		\item \cdtIdRef{MSG7}{Registro repetido}: Se muestra en la pantalla \IUref{IU2.1}{Registrar proyecto} cuando se registre un proyecto con un nombre o clave que ya exista.
		\item \cdtIdRef{MSG12}{Ha ocurrido un error}: Se muestra en la pantalla \IUref{IU2}{Gestionar proyectos de Administrador} cuando no exista información de los estados de un proyecto.
		\item \cdtIdRef{MSG17}{Falta información}: Se muestra en la pantalla \IUref{IU2}{Gestionar proyectos de Administrador} cuando no existan colaboradores registrados.
		\item \cdtIdRef{MSG26}{Orden de fechas}: Se muestra en la pantalla \IUref{IU2.1}{Registrar proyecto} cuando el actor ingrese fechas de término que no son posteriores a las fechas
		de inicio correspondientes.
		\end{itemize}
		}
		\UCitem{Tipo}{Secundario, extiende del caso de uso \UCref{CU2}{Gestionar proyectos de Administrador}}
	\end{UseCase}
%--------------------------------------
	\begin{UCtrayectoria}
		\UCpaso[\UCactor] Solicita registrar un proyecto oprimiendo el botón \IUbutton{Registrar} de la pantalla \IUref{IU2}{Gestionar proyectos de Administrador}.
		\UCpaso[\UCsist] Verifica que exista información referente a los estados de un proyecto, con base en la regla de negocio \BRref{RN20}{Verificación de catálogos}. \Trayref{A}
		\UCpaso[\UCsist] Verifica que exista al menos un colaborador, con base en la regla de negocio \BRref{RN20}{Verificación de catálogos}. \Trayref{B}
		\UCpaso[\UCsist] Muestra la pantalla \IUref{IU2.1}{Registrar proyecto}.
		\UCpaso[\UCactor] Ingresa la información solicitada en la pantalla.
		\UCpaso[\UCactor] Solicita guardar el proyecto oprimiendo el botón \IUbutton{Aceptar} de la pantalla \IUref{IU2.1}{Registrar proyecto}. \Trayref{C}
		\UCpaso[\UCsist] Verifica que el actor ingrese todos los campos obligatorios con base en la regla de negocio \BRref{RN8}{Datos obligatorios}. \Trayref{D}
		\UCpaso[\UCsist] Verifica que el nombre del proyecto no se encuentre registrado en el sistema con base en la regla de negocio \BRref{RN6}{Unicidad de nombres}. \Trayref{E}
		\UCpaso[\UCsist] Verifica que la clave del proyecto no se encuentre registrada en el sistema con base en la regla de negocio \BRref{RN22}{Unicidad de la clave del Proyecto}. \Trayref{F}
		\UCpaso[\UCsist] Verifica que los datos requeridos sean proporcionados correctamente con base en la regla de negocio \BRref{RN7}{Información correcta}. \Trayref{G} \Trayref{H}
		\UCpaso[\UCsist] Verifica que la fecha de término sea posterior a la fecha de inicio. \Trayref{I}
		\UCpaso[\UCsist] Verifica que la fecha de término programada sea posterior a la fecha de inicio programada. \Trayref{J}
		\UCpaso[\UCsist] Registra la información del proyecto en el sistema
		\UCpaso[\UCsist] Muestra el mensaje \cdtIdRef{MSG1}{Operación exitosa} en la pantalla \IUref{IU2}{Gestionar proyectos de Administrador} para indicar al actor que el registro se ha realizado exitosamente.
	\end{UCtrayectoria}		
%--------------------------------------
		\begin{UCtrayectoriaA}{A}{No hay estados con los que puede iniciar un proyecto}
			\UCpaso[\UCsist] Muestra el mensaje \cdtIdRef{MSG12}{Ha ocurrido un error} en la pantalla \IUref{IU2}{Gestionar proyectos de Administrador} para indicar que no es posible realizar la operación debido a la falta de información necesaria para el sistema.
		\end{UCtrayectoriaA}

%--------------------------------------

\subsubsection{Puntos de extensión}

\UCExtenssionPoint{El actor requiere registrar un proyecto.}{Paso \ref{P4} de la trayectoria principal.}{\UCref{CU2.1}{Registrar Proyecto}}
\UCExtenssionPoint{El actor requiere modificar un proyecto.}{Paso \ref{P4} de la trayectoria principal.}{\UCref{CU2.2}{Modificar Proyecto}}
\UCExtenssionPoint{El actor requiere eliminar un proyecto.}{Paso \ref{P4} de la trayectoria principal.}{\UCref{CU2.3}{Eliminar Proyecto}}
