	\begin{UseCase}{CU11.1}{Registrar Caso de uso}{
			Este caso de uso permite al actor registrar un caso de uso. Al ingresar los datos, el actor podrá utilizar un token que le desplegará una lista de elementos disponibles para su utilización.
	}
		\UCitem{Versión}{\color{Gray}0.1}
		\UCitem{Actor}{\hyperlink{jefe}{Líder de Análisis}, \hyperlink{analista}{Analista}}
		\UCitem{Propósito}{Registrar la información de un caso de uso.}
		\UCitem{Entradas}{
		\begin{itemize}
			\item De la sección Información de un caso de uso.
			\begin{itemize}
				\item \cdtRef{casoUso:numeroCU}{Número}: Se escribe desde el teclado. 
				\item \cdtRef{casoUso:nombreCU}{Nombre}: Se escribe desde el teclado.
				\item \cdtRef{casoUso:resumenCU}{Resumen}: Se escribe desde el teclado.
			\end{itemize}
			\item De la sección Descripción del caso de uso:
			\begin{itemize}
				\item \cdtRef{actoreEntidad}{Actores}: Se escribe una palabra y se selecciona una sugerencia de la lista.
				\item Entradas: Se escribe una palabra y se selecciona una sugerencia de la lista.
				\item Salidas: Se escribe una palabra y se selecciona una sugerencia de la lista.
				\item \cdtRef{BREntidad}{Reglas de Negocio}: Se escribe una palabra y se selecciona una sugerencia de la lista.
			\end{itemize}
		\end{itemize}	
		}
		\UCitem{Salidas}{\begin{itemize}
				\item \cdtRef{proyectoEntidad:claveProyecto}{Clave del proyecto}: Lo obtiene el sistema.
				\item \cdtRef{proyectoEntidad:nombreProyecto}{Nombre del proyecto}: Lo obtiene el sistema.
				\item \cdtRef{moduloEntidad:claveModulo}{Clave del Módulo}: Lo obtiene el sistema.
				\item \cdtRef{moduloEntidad:nombreModulo}{Nombre del Módulo}: Lo obtiene el sistema.
				\item \cdtRef{casoUso:claveCU}{Clave}: Lo calcula el sistema mediante la regla de negocio \BRref{RN12}{Identificador de elemento}.
				\item \cdtIdRef{MSG1}{Operación exitosa}: Se muestra en la pantalla \IUref{IU6}{Gestionar Casos de uso} para indicar que el registro fue exitoso.
		\end{itemize}}
		\UCitem{Destino}{Pantalla}
		\UCitem{Precondiciones}{
			\begin{itemize}
				\item Que exista al menos una regla de negocio registrada en el proyecto actual.
				\item Que exista al menos una entidad registrada en el proyecto actual.
				\item Que exista al menos un atributo de una entidad registrada en el proyecto actual.
				\item Que exista al menos un caso de uso registrado en el proyecto actual.
				\item Que exista al menos una pantalla registrada en el proyecto actual.
				\item Que exista al menos una acción de una pantalla registrada en el proyecto actual.
				\item Que exista al menos un mensaje registrado en el proyecto actual.
				\item Que exista al menos un actor registrado en el proyecto actual.
				\item Que exista al menos un término de glosario registrado en el proyecto actual.
			\end{itemize}
		}
		\UCitem{Postcondiciones}{
		\begin{itemize}
			\item Se registrará un nuevo caso de uso en el sistema con estado ''Edición''.
		\end{itemize}
		}
		\UCitem{Errores}{\begin{itemize}
		\item \cdtIdRef{MSG4}{Dato obligatorio}: Se muestra en la pantalla \IUref{IU6.1}{Registrar Caso de uso} cuando no se ha ingresado un dato marcado como obligatorio.
		\item \cdtIdRef{MSG29}{Formato incorrecto}: Se muestra en la pantalla \IUref{IU6.1}{Registrar Caso de uso} cuando el tipo de dato ingresado no cumple con el tipo de dato solicitado en el campo.
		\item \cdtIdRef{MSG5}{Formato de campo incorrecto}: Se muestra en la pantalla \IUref{IU6.1}{Registrar Mensaje} cuando el número del caso de uso contiene un carácter no válido.
		\item \cdtIdRef{MSG6}{Longitud inválida}: Se muestra en la pantalla \IUref{IU6.1}{Registrar Caso de uso} cuando se ha excedido la longitud de alguno de los campos.
		\item \cdtIdRef{MSG7}{Registro repetido}: Se muestra en la pantalla \IUref{IU6.1}{Registrar Caso de uso} cuando se registre un actor con un nombre que ya se encuentre registrado en el sistema.
		\item \cdtIdRef{MSG14}{Dato no registrado}: Se muestra en la pantalla \IUref{IU6.1}{Registrar Caso de uso} cuando un elemento referenciado no existe en el sistema.
		\item \cdtIdRef{MSG19}{Token Incorrecto}: Se muestra en la pantalla \IUref{IU6.1}{Registrar Caso de uso} cuando el token ingresado se encuentra estructurado de forma incorrecta.
		\end{itemize}.
		}
		\UCitem{Tipo}{Secundario, extiende del caso de uso \UCref{CU11}{Gestionar Casos de uso}.}
	\end{UseCase}
%--------------------------------------
	\begin{UCtrayectoria}
		\UCpaso[\UCactor] Solicita registrar un caso de uso oprimiendo el botón \IUbutton{Registrar} de la pantalla \IUref{IU6}{Gestionar Casos de uso}.
		\UCpaso[\UCsist] Obtiene las reglas de negocio del proyecto actual registradas en el sistema.
		\UCpaso[\UCsist] Obtiene las entidades del proyecto actual registradas en el sistema.
		\UCpaso[\UCsist] Obtiene los atributos del proyecto actual registradas en el sistema.
		\UCpaso[\UCsist] Obtiene los casos de uso del proyecto actual registradas en el sistema.
		\UCpaso[\UCsist] Obtiene las pantallas del proyecto actual registradas en el sistema.
		\UCpaso[\UCsist] Obtiene las acciones de las pantallas del proyecto actual registradas en el sistema.
		\UCpaso[\UCsist] Obtiene los mensajes del proyecto actual registradas en el sistema.
		\UCpaso[\UCsist] Obtiene los actores del proyecto actual registradas en el sistema.
		\UCpaso[\UCsist] Obtiene los términos de glosario del proyecto actual registradas en el sistema.
		\UCpaso[\UCsist] Muestra la pantalla \IUref{IU6.1}{Registrar Caso de uso}.
		\UCpaso[\UCactor] Ingresa la información general del caso de uso.
		\UCpaso[\UCactor] Ingresa los actores del caso de uso.
		\UCpaso[\UCactor] Solicita guardar la información del actor oprimiendo el botón \IUbutton{Aceptar} de la pantalla \IUref{IU6.1}{Registrar Caso de uso}. \label{CU10.1-P5} \Trayref{RACT-B} 
		\UCpaso[\UCsist] Verifica que el actor ingrese todos los campos obligatorios con base en la regla de negocio \BRref{RN8}{Datos obligatorios}. \Trayref{RACT-C}
		\UCpaso[\UCsist] Verifica que los datos requeridos sean proporcionados correctamente con base en la regla de negocio \BRref{RN7}{Información correcta}. \Trayref{RACT-D} \Trayref{RACT-E}
		\UCpaso[\UCsist] Verifica que el nombre del actor no se encuentre registrado en el sistema con base en la regla de negocio \BRref{RN6}{Unicidad de nombres}. \Trayref{RACT-F} 
		\UCpaso[\UCsist] Registra la información del actor en el sistema.
		\UCpaso[\UCsist] Muestra el mensaje \cdtIdRef{MSG1}{Operación exitosa} en la pantalla \IUref{IU8}{Gestionar Actores} para indicar al actor que el registro se ha realizado exitosamente.
	\end{UCtrayectoria}		
%--------------------------------------
	
	\begin{UCtrayectoriaA}{RACT-A}{No existe información en los catálogos.}
		\UCpaso[\UCactor] Muestra el mensaje \cdtIdRef{MSG12}{Ha ocurrido un error} en la pantalla \IUref{IU8}{Gestionar Actores}.
	\end{UCtrayectoriaA}

	\begin{UCtrayectoriaA}{RACT-B}{El actor desea cancelar la operación.}
		\UCpaso[\UCactor] Solicita cancelar la operación oprimiendo el botón \IUbutton{Cancelar} de la pantalla \IUref{IU6.1}{Registrar Caso de uso}.
		\UCpaso[\UCsist] Muestra la pantalla \IUref{IU8}{Gestionar Actores}.
	\end{UCtrayectoriaA}

	\begin{UCtrayectoriaA}{RACT-C}{El actor no ingresó algún dato marcado como obligatorio.}
		\UCpaso[\UCsist] Muestra el mensaje \cdtIdRef{MSG4}{Dato obligatorio} y señala el campo que presenta el error en la pantalla \IUref{IU6.1}{Registrar Caso de uso}, indicando al actor que el dato es obligatorio.
		\UCpaso Regresa al paso \ref{CU10.1-P4} de la trayectoria principal.
	\end{UCtrayectoriaA}

	\begin{UCtrayectoriaA}{RACT-D}{El actor proporciona un dato que excede la longitud máxima.}
		\UCpaso[\UCsist] Muestra el mensaje \cdtIdRef{MSG6}{Longitud inválida} y señala el campo que excede la longitud en la pantalla \IUref{IU6.1}{Registrar Caso de uso}, para indicar que el dato excede el tamaño máximo permitido.
		\UCpaso Regresa al paso \ref{CU10.1-P4} de la trayectoria principal.
	\end{UCtrayectoriaA}

	\begin{UCtrayectoriaA}{RMSG-E}{El actor ingresó un número de mensaje con un tipo de dato incorrecto.}
		\UCpaso[\UCsist] Muestra el mensaje \cdtIdRef{MSG5}{Formato de campo incorrecto} y señala el campo que presenta el dato inválido en la pantalla \IUref{IU10.1}{Registrar Mensaje}, para indicar que se ha ingresado un tipo de dato inválido.
		\UCpaso Regresa al paso \ref{CU10.1-P4} de la trayectoria principal.
	\end{UCtrayectoriaA}

	\begin{UCtrayectoriaA}{RACT-F}{El actor ingresó un tipo de dato incorrecto.}
		\UCpaso[\UCsist] Muestra el mensaje \cdtIdRef{MSG29}{Formato incorrecto} y señala el campo que presenta el dato inválido en la pantalla \IUref{IU6.1}{Registrar Caso de uso}, para indicar que se ha ingresado un tipo de dato inválido.
		\UCpaso Regresa al paso \ref{CU10.1-P4} de la trayectoria principal.
	\end{UCtrayectoriaA}
	
	\begin{UCtrayectoriaA}{RACT-G}{El actor ingresó un nombre de un actor repetido.}
		\UCpaso[\UCsist] Muestra el mensaje \cdtIdRef{MSG7}{Registro repetido} y señala el campo que presenta la duplicidad en la pantalla \IUref{IU6.1}{Registrar Caso de uso}, indicando al actor que existe un mensaje con el mismo nombre.
		\UCpaso Regresa al paso \ref{CU10.1-P4} de la trayectoria principal.
	\end{UCtrayectoriaA}
