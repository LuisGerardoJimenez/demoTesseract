\section{Términos del Negocio}
\label{sec:terminosDeNegocio}

El presente glosario presenta los términos utilizados a lo largo del documento y tiene como finalidad establecer el lenguaje base que permita comprender la especificación del sistema.

\begin{description}
	% Ejemplo de un término literal.
	\item[\hypertarget{tAtributo}{Atributo:}] Son las características que definen o identifican a una entidad en un conjunto de entidades. 
	% Ejemplo de un término de entidad
	\item[\hypertarget{tArchivoDigital}{Archivo Digital:}] Equivalente digital de los archivos escritos en libros, tarjetas, libretas, papel o microfichas del entorno de oficina tradicional.
	
	\item[\hypertarget{tBooleano}{Boooleano:}] Es un \hyperlink{tTipoDato}{tipo de dato} que puede tomar los siguientes valores: verdadero ó falso (1 ó 0).	
	\item[\hypertarget{tCadena}{Cadena:}] Es el \hyperlink{tTipoDato}{tipo de dato} definido por cualquier valor que se compone de una secuencia de caracteres, con o sin acentos, espacios, dígitos y signos de puntuación. Existen tres tipos de cadenas: palabra, frase y párrafo.
	
	\item[\hypertarget{tCardinalidad}{Cardinalidad:}] Es el número de actores que participarán o serán requeridos en el sistema. Es un \hyperlink{tTipoDato}{tipo de dato} para el sistema y puede tomar alguno de los siguientes valores: Uno, Muchos u Otro.
	
	\item[\hypertarget{tElemento}{Elemento:}] Se utiliza para referirse a los casos de uso, pantallas, reglas de negocio, entidades, término del glosario, mensajes y actores. 

	\item[\hypertarget{tEntero}{Entero:}] Es el \hyperlink{tTipoDato}{tipo de dato} \hyperlink{tNumerico}{numérico} definido por todos los valores numéricos enteros, tanto positivos como negativos.
	
	\item[\hypertarget{tEntidad}{Entidad:}] Término genérico que se utiliza para determinar un ente el cual puede ser concreto, abstracto o conceptual por ejemplo: Caso de uso, proyecto, módulo, etc. La entidades se caracterizan con atributos que la definen.
	
	\item[\hypertarget{tEdoElem}{Estado del Elemento:}] Es un identificador que indica la situación de un elemento. Es un \hyperlink{tTipoDato}{tipo de dato} para el sistema y puede tomar alguno de los siguientes valores: Pre-registro, Edición, Terminado, Pendiente de Corrección, Por Liberar o Liberado.
	
	\item[\hypertarget{tEdoProy}{Estado del Proyecto:}] Es un identificador que indica la situación de un proyecto. Es un \hyperlink{tTipoDato}{tipo de dato} para el sistema y puede tomar alguno de los siguientes valores: En Negociación, Iniciado o Terminado.
	
	\item[\hypertarget{tFecha}{Fecha:}] Es un \hyperlink{tTipoDato}{tipo de dato} que indica un día único en referencia al calendario gregoriano. Los tipos de fecha utilizados son: \hyperlink{tFechaCorta}{fecha corta} y \hyperlink{tFechaLarga}{fecha larga}.
	
	\item[\hypertarget{tFechaCorta}{Fecha Corta:}] Es la representación del \hyperlink{tTipoDato}{tipo de dato} \hyperlink{tFecha}{Fecha} en la forma DD/MM/YYYY, por ejemplo: 16/05/2017.
	
	\item[\hypertarget{tFechaInitPro}{Fecha de inicio del proyecto:}] Es la \hyperlink{tFecha}{fecha} de inicio el proceso de software del sistema.
	
	\item[\hypertarget{tFechaLarga}{Fecha Larga:}] Es la representación del \hyperlink{tTipoDato}{tipo de dato} \hyperlink{tFecha}{fecha} en la forma DD de MM del YYYY, por ejemplo: 16 de mayo del 2017.
	
	\item[\hypertarget{tFrase}{Frase:}] Es un \hyperlink{tTipoDato}{tipo de dato} conformado por \hyperlink{tPalabra}{palabras} y espacios.
	
	\item[\hypertarget{tImagen}{Imagen:}] Es una imagen de tamaño pequeño (no más de un Megabyte) en formato jpeg.
	
	\item[\hypertarget{tNumerico}{Númerico:}] Es un \hyperlink{tTipoDato}{tipo de dato} que se compone de la combinación de los símbolos {\em 0, 1, 2, 3, 4, 5, 6, 7, 8, 9, . y -}, que expresan una cantidad en relación a su unidad.
	
	\item[\hypertarget{tOpcional}{Opcional:}] Es un elemento que el actor puede o no proporcionar en el formulario o la pantalla, su decisión no afectará la ejecución de la operación solicitada.
	
	\item[\hypertarget{tPalabra}{Palabra:}] Es un \hyperlink{tTipoDato}{tipo de dato} \hyperlink{tCadena}{cadena} conformado por el alfabeto y símbolos especiales como son {\em \#, -,
	\$, \%, \&, (,), etc.} y se caracteriza por no tener espacios.
	
	\item[\hypertarget{tParametroM}{Parámetro del Mensaje:}] Es una palabra que se solicita en un mensaje parametrizado. Es un \hyperlink{tTipoDato}{tipo de dato} para el sistema y puede tomar alguno de los siguientes valores: Determinado, Indeterminado, Operación, Atributo, Entidad, Regla de negocio, entre otros.
	
	\item[\hypertarget{tParametroP}{Parámetro del Paso:}] Es un elemento que se solicita en un paso. Es un \hyperlink{tTipoDato}{tipo de dato} para el sistema y puede tomar alguno de los siguientes valores: Atributo, Casos de Uso, Pantalla, Regla de Negocio, Entidad, Término del Glosario, Mensaje, Actor, Paso o Acción.
	
	\item[\hypertarget{tParrafo}{Párrafo:}] Es un \hyperlink{tTipoDato}{tipo de dato} conformado por \hyperlink{tFrase}{frases}.
	
	\item[\hypertarget{tRequerido}{Requerido:}] Es un \hyperlink{tTipoDato}{tipo de dato} que debe proporcionarse de manera obligatoria. La ejecución de la operación solicitada dependerá de que se proporcione este dato.
	
	\item[\hypertarget{tRol}{Rol:}] Es la función de un colaborador dentro de un proyecto. Es un \hyperlink{tTipoDato}{tipo de dato} para el sistema y puede tomar alguno de los siguientes valores: Analista o Líder de Análisis.
	
	\item[\hypertarget{tSeccion}{Sección:}] Es el área del caso de uso que se revisa y sobre la que se hacen observaciones. Es un \hyperlink{tTipoDato}{tipo de dato} para el sistema y puede tomar alguno de los siguientes valores: Información General, Descripción, Precondiciones, Postcondiciones, Trayectorias o Puntos de extensión.
	
	\item[\hypertarget{tTelefono}{Teléfono:}] Secuencia de dígitos utilizada para identificar una línea telefónica. Es un \hyperlink{tTipoDato}{tipo de dato} para el sistema.
	
	\item[\hypertarget{tTipoAcc}{Tipo de Acción:}] Es un elemento que permite solicitar una operación desde la pantalla. Es un \hyperlink{tTipoDato}{tipo de dato} para el sistema y puede tomar alguno de los siguientes valores: Botón, Liga, Opción del Menú, entre otros.
	
	\item[\hypertarget{tTipoDato}{Tipo de Dato:}] Es el dominio o conjunto de valores que puede tomar un atributo de una entidad en el modelo de información. Los tipos de datos utilizados son: \hyperlink{tPalabra}{palabra}, \hyperlink{tFrase}{frase}, \hyperlink{tParrafo}{párrafo}, \hyperlink{tNumerico}{numérico}, \hyperlink{tFecha}{fecha} y \hyperlink{tBooleano}{booleano}.
	
	\item[\hypertarget{tTipoDatoP}{Tipo de Dato(sistema):}] Es el \hyperlink{tTipoDato}{tipo de dato} que puede tener un atributo. Es un \hyperlink{tTipoDato}{tipo de dato} para el sistema y puede tomar alguno de los siguientes valores: Entero, Flotante, Booleano, Cadena o Fecha.
	
	\item[\hypertarget{tTipoRN}{Tipo de Regla de Negocio:}] Es la categoría a la que pertenece una regla de negocio de acuerdo a las descritas en los requerimientos del sistema. Es un  para el sistema y puede tomar alguno de los siguientes valores: Verificación de catálogos, Operaciones aritméticas, Unicidad de parámetros, Datos obligatorios, Longitud correcta, Tipo de dato correcto, Formato de archivos, Tamaño de archivos, Intervalo de fechas correctas, Formato correcto u Otro.
	
\end{description}